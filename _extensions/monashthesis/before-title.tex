%% CAPTIONS
\usepackage{caption}  % Load the caption package to customize figure and table captions.
\DeclareCaptionStyle{italic}[justification=centering] % Define a caption style that centers the text and uses italic font.
 {labelfont={bf},textfont={it},labelsep=colon}  % Set bold font for labels, italic font for text, and use a colon as a separator.
\captionsetup[figure]{style=italic,format=hang,singlelinecheck=true}  % Apply the italic style to figure captions, with hanging indentation and single-line checks.
\captionsetup[table]{style=italic,format=hang,singlelinecheck=true}  % Apply the same settings to table captions.

%% FONT
\usepackage{bera}  % Load the Bera font, used for sans-serif text.
\usepackage[charter]{mathdesign}  % Use the Charter font for text and math symbols.
\usepackage[scale=0.9]{sourcecodepro}  % Load Source Code Pro font for monospace text, scaled down to 90%.
\usepackage[lf,t]{FiraSans}  % Use Fira Sans font for sans-serif text with lining figures and tabular alignment.
\usepackage{fontawesome}  % Load the FontAwesome icons for use in the document.

%% HEADERS AND FOOTERS
\usepackage{fancyhdr}  % Load the fancyhdr package to customize headers and footers.
\pagestyle{fancy}  % Apply a fancy page style to the entire document.
\rfoot{\Large\sffamily\raisebox{-0.1cm}{\textbf{\thepage}}}  % Set the right footer to show the page number in large sans-serif text.
\makeatletter  % Begin a LaTeX programming section.
\lhead{\textsf{\expandafter{\@title}}}  % Set the left header to the document title in sans-serif text.
\makeatother  % End the LaTeX programming section.
\rhead{}  % Clear the right header.
\cfoot{}  % Clear the center footer.
\setlength{\headheight}{15pt}  % Set the height of the header to 15pt.
\renewcommand{\headrulewidth}{0.4pt}  % Set the thickness of the header rule (line) to 0.4pt.
\renewcommand{\footrulewidth}{0.4pt}  % Set the thickness of the footer rule to 0.4pt.
\fancypagestyle{plain}{%  % Define a fancy page style for plain pages (e.g., chapter start pages).
\fancyhf{} % Clear all header and footer fields.
\fancyfoot[C]{\sffamily\thepage} % Set the center footer to the page number in sans-serif text.
\renewcommand{\headrulewidth}{0pt}  % Remove the header rule for plain pages.
\renewcommand{\footrulewidth}{0pt}}  % Remove the footer rule for plain pages.

%% MATHS
\usepackage{bm,amsmath}  % Load packages for bold math symbols and advanced math formatting.
\allowdisplaybreaks  % Allow page breaks in multi-line equations.

%% GRAPHICS
\makeatletter  % Begin a LaTeX programming section.
\def\fps@figure{htbp}  % Set the default figure placement to here, top, bottom, or page.
\makeatother  % End the LaTeX programming section.
\setcounter{topnumber}{2}  % Allow up to 2 floats at the top of a page.
\setcounter{bottomnumber}{2}  % Allow up to 2 floats at the bottom of a page.
\setcounter{totalnumber}{4}  % Allow up to 4 floats per page.
\renewcommand{\topfraction}{0.85}  % Allow floats to take up to 85% of the top of the page.
\renewcommand{\bottomfraction}{0.85}  % Allow floats to take up to 85% of the bottom of the page.
\renewcommand{\textfraction}{0.15}  % Require at least 15% of a page to be text (not floats).
\renewcommand{\floatpagefraction}{0.8}  % Allow floats to occupy up to 80% of a float-only page.
\graphicspath{{figures/}}  % Set the directory path for graphics files.

%% SECTION TITLES
\usepackage[compact,sf,bf]{titlesec}  % Load the titlesec package with compact, sans-serif, and bold section title styles.
\titleformat*{\section}{\Large\sf\bfseries}  % Set the format for \section titles to large, sans-serif, and bold.
\titleformat*{\subsection}{\large\sf\bfseries}  % Set the format for \subsection titles to large, sans-serif, and bold.
\titleformat*{\subsubsection}{\sf\bfseries}  % Set the format for \subsubsection titles to sans-serif and bold.
\titlespacing{\section}{0pt}{*5}{*1}  % Set section title spacing: no indent, 5 lines before, 1 line after.
\titlespacing{\subsection}{0pt}{*2}{*0.2}  % Set subsection title spacing: no indent, 2 lines before, 0.2 lines after.
\titlespacing{\subsubsection}{0pt}{*1}{*0.1}  % Set subsubsection title spacing: no indent, 1 line before, 0.1 lines after.

%% TABLES
\usepackage{booktabs,tabu}  % Load the booktabs and tabu packages for advanced table formatting.
\usepackage{graphicx}  % Load the graphicx package to include graphics and resize boxes.
\usepackage{tabularx}  % Load the tabularx package for tables with automatically adjusting column widths.

% Custom table environment to automatically resize tables
\newenvironment{resizetable}  % Define a new environment called resizetable.
  {\begin{table}\resizebox{\textwidth}{!}{\begin{tabular}}  % Start a table and resize it to fit the text width.
  {\end{tabular}}\end{table}}  % End the tabular environment and table.

% Custom tabular environment that uses tabularx
\newenvironment{xtable}  % Define a new environment called xtable.
  {\begin{table}\centering\begin{tabularx}{\textwidth}{lXX}} % Start a table, centered, with lXX columns in tabularx.
  {\end{tabularx}\end{table}}  % End the tabularx environment and table.

%% BIBLIOGRAPHY.
\makeatletter  % Begin a LaTeX programming section.
\@ifpackageloaded{biblatex}{  % Check if the biblatex package is loaded.
\ExecuteBibliographyOptions{bibencoding=utf8,minnames=1,maxnames=3, maxbibnames=99,dashed=false,terseinits=true,giveninits=true,uniquename=false,uniquelist=false,doi=false, isbn=false,url=true,sortcites=false}  % Set various bibliography options.
\DeclareFieldFormat{url}{\texttt{\url{#1}}}  % Format URLs as monospace text with the \url command.
\DeclareFieldFormat[article]{pages}{#1}  % Format page numbers in articles.
\DeclareFieldFormat[inproceedings]{pages}{\lowercase{pp.}#1}  % Format page ranges in proceedings with lowercase "pp.".
\DeclareFieldFormat[incollection]{pages}{\lowercase{pp.}#1}  % Format page ranges in collections with lowercase "pp.".
\DeclareFieldFormat[article]{volume}{\mkbibbold{#1}}  % Format volume numbers in articles as bold.
\DeclareFieldFormat[article]{number}{\mkbibparens{#1}}  % Format issue numbers in articles in parentheses.
\DeclareFieldFormat[article]{title}{\MakeCapital{#1}}  % Capitalize the first word of article titles.
\DeclareFieldFormat[article]{url}{}  % Remove URLs from article references.
\DeclareFieldFormat[inproceedings]{title}{#1}  % Format titles in proceedings.
\DeclareFieldFormat{shorthandwidth}{#1}  % Set the width for shorthand entries.
\usepackage{xpatch}  % Load the xpatch package for patching LaTeX commands.
\xpatchbibmacro{volume+number+eid}{\setunit*{\adddot}}{}{}{}  % Patch the bibliography macro to remove the dot after the volume+number+eid unit.
% Remove In: for an article.
\renewbibmacro{in:}{%  % Redefine the "in:" macro.
  \ifentrytype{article}{}{%
  \printtext{\bibstring{in}\intitlepunct}}}  % If the entry is not an article, print "In:" followed by the title punctuation.
\AtEveryBibitem{\clearfield{month}}  % Clear the month field from all bibliography items.
\AtEveryCitekey{\clearfield{month}}  % Clear the month field from all citations.
\DeclareDelimFormat[cbx@textcite]{nameyeardelim}{\addspace}  % Define a delimiter format for name-year citations.
\renewcommand*{\finalnamedelim}{\addspace\&\space}  % Use " & " as the final name delimiter in citations.
}{}  % End the conditional check for biblatex.
\makeatother  % End the LaTeX programming section.

% HYPERLINK SETTINGS
\hypersetup{
     pdfcreator={Quarto -> pandoc -> LaTeX -> pdf}  % Set metadata for PDF creation.
}

%% PAGE BREAKING to avoid widows and orphans
\clubpenalty = 2000  % Increase the penalty for breaking a page after the first line of a paragraph (widows).
\widowpenalty = 2000  % Increase the penalty for breaking a page before the last line of a paragraph (orphans).
\usepackage{microtype}  % Load the microtype package for better typographic quality.

% Custom command for inserting a centered title page
\newcommand{\insertchaptertitlepage}[1]{  % Define a new command for inserting a title page.
  \clearpage  % Start a new page.
  \begin{titlepage}  % Begin the title page environment.
    \centering  % Center all content on the title page.
    \vspace*{\fill}  % Add vertical space before the title.
    {\Huge\bfseries #1\par}  % Print the title in huge, bold font.
    \vspace*{\fill}  % Add vertical space after the title.
  \end{titlepage}  % End the title page environment.
}


